% \omiss produces '[...]'
\newcommand{\omissis}{[\dots\negthinspace]}

% Itemize symbols
% see: https://tex.stackexchange.com/a/62497
% \renewcommand{\labelitemi}{$\bullet$}
% \renewcommand{\labelitemii}{$\cdot$}
% \renewcommand{\labelitemiii}{$\diamond$}
% \renewcommand{\labelitemiv}{$\ast$}

% Custom hyphenation rules
\hyphenation {
    e-sem-pio
    ex-am-ple
}

% Images path
\graphicspath{{images/}}

% Page format settings
% see: http://wwwcdf.pd.infn.it/AppuntiLinux/a2547.htm
\setlength{\parindent}{14pt}    % first row indentation
\setlength{\parskip}{0pt}       % paragraphs spacing

% Load variables
\newcommand{\myName}{Marco Brugin}
\newcommand{\myTitle}{Creazione di una Data Pipeline per il trattamento dei dati con Apache Kafka e Apache Druid}
\newcommand{\myDegree}{Tesi di laurea triennale}
\newcommand{\myUni}{Università degli Studi di Padova}
\newcommand{\myFaculty}{Corso di Laurea in Informatica}
\newcommand{\myDepartment}{Dipartimento di Matematica ``Tullio Levi-Civita''}
\newcommand{\profTitle}{Prof.}
\newcommand{\myProf}{Ombretta Gaggi}
\newcommand{\myLocation}{Padova}
\newcommand{\myAA}{2022-2023}
\newcommand{\myTime}{22 Settebre}
\newcommand{\myMatricola}{2010012}
% PDF file metadata fields
% when updating them delete the build directory, otherwise they won't change
\RequirePackage{filecontents}
\begin{filecontents*}{\jobname.xmpdata}
  \Title{Document's title}
  \Author{Author's name}
  \Language{it-IT}
  \Subject{Short description}
  \Keywords{keyword1\sep keyword2\sep keyword3}
\end{filecontents*}


% Acronyms
\newacronym[description={\glslink{ohsg}{Occupational Health and Safety}}]
    {ohs}{OH\&S}{Occupational Health and Safety}
\newacronym[description={\glslink{sgsig}{Sistema Gestione Sicurezza Informazioni}}]
    {sgsi}{SGSI}{Sistema Gestione Sicurezza Informazioni}
\newacronym[description={\glslink{ictg}{Information and Communication Technology}}]
    {ict}{ICT}{Information and Communication Technology}
\newacronym[description={\glslink{olapg}{Online Analytical Processing}}]
    {olap}{OLAP}{Online Analytical Processing}
\newglossaryentry{olapg} {
    name=\glslink{olap}{OLAP},
    text=Online Analytical Processing,
    sort=olap,
    description={L'Online Analytical Processing è un insieme di metodi finalizzato a effettuare analisi rapide e approfondite su grandi volumi di dati, provenienti da uno o piu sorgenti, per prendere decisioni a riguardo}
}
\newglossaryentry{ictg} {
    name=\glslink{ict}{ICT},
    text=Information and Communication Technology,
    sort=ict,
    description={ICT è un termine generico che indica tutte le tecnologie che riguardano la trasmissione, la ricezione e l'elaborazione di informazioni sotto forma di segnali elettronici o elettromagnetici}
}
\newglossaryentry{sgsig} {
    name=\glslink{sgsi}{SGSI},
    text=Sistema Gestione Sicurezza Informazioni,
    sort=sgsi,
    description={Lo SGSI è un sistema di gestione che permette di gestire in modo strutturato la sicurezza delle informazioni aziendali}
}
\newglossaryentry{ohsg} {
    name=\glslink{ohs}{OH\&S},
    text=Occupational Health and Safety,
    sort=OH\&S,
    description={L'Occupational Health and Safety è un sistema di gestione che permette di gestire in modo strutturato la salute e sicurezza dei lavoratori}
}
\newglossaryentry{Data Pipeline}{
    name= \glslink{Data Pipeline}{Data Pipeline},
    text=Data Pipeline,
    sort=Data Pipeline,
    description={Una Data Pipeline è un insieme di operazioni che permettono di trasformare e analizzare i dati in modo da renderli pronti per l'archiviazione}
}
\newglossaryentry{Data Processing}
{
    name=\glslink{Data Processing}{Data Processing},
    text=Data Processing,
    sort=Data Processing,
    description={Il Data Processing (elaborazione dati) si riferisce alla manipolazione, trasformazione e analisi di dati grezzi al fine di ottenere informazioni significative e approfondite. Comprende una serie di passaggi che permettono di convertire dati non strutturati in forme più utili e che facilitano la loro elaborazione}
}
\newglossaryentry{Data Analytics}
{
    name=\glslink{Data Analytics}{Data Analytics},
    text=Data Analytics,
    sort=Data Analytics,
    description={Il Data Analytics (analisi dati) è il processo di esaminare i dati per trarne conclusioni sull'informazione che contengono}
}
\newglossaryentry{Business Innovation}
{
    name=\glslink{Business Innovation}{Business Innovation},
    text=Business Innovation,
    sort=Business Innovation,
    description={La Business Innovation è un processo che permette d' introdurre nuovi metodi, idee, prodotti e servizi per migliorare l'efficienza, la produttività e la competitività di un'organizzazione}
}
\newglossaryentry{Middleware}
{
    name=\glslink{Middleware}{Middleware},
    text=Middleware,
    sort=Middleware,
    description={Il Middleware è un software che si interpone tra un sistema operativo e le applicazioni che vengono eseguite al di sopra. Il suo scopo è quello di facilitare lo sviluppo di applicazioni e di nascondere la complessità del sistema operativo sottostante}
}
\newglossaryentry{repository}
{
    name=\glslink{repository}{repository},
    text=repository,
    sort=repository,
    description={Un repository è un ambiente di archiviazione centralizzato in cui vengono conservati e gestiti i dati}
}
\newglossaryentry{metadati}
{
    name=\glslink{metadati}{metadati},
    text=metadati,
    sort=metadati,
    description={I metadati sono dati che descrivono altri dati. Sono utilizzati per descrivere le caratteristiche dei dati e per facilitarne la ricerca e l'organizzazione}
}
\newglossaryentry{working directory}
{
    name=\glslink{working directory}{working directory},
    text=working directory,
    sort=working directory,
    description={La working directory è la directory di lavoro corrente}
}
\newglossaryentry{Scrum}
{
    name=\glslink{Scrum}{Scrum},
    text=Scrum,
    sort=Scrum,
    description={Scrum è un framework agile per la gestione del ciclo di sviluppo del software}
}
\newglossaryentry{modello incrementale}
{
    name=\glslink{modello incrementale}{modello incrementale},
    text=modello incrementale,
    sort=modello incrementale,
    description={Il modello incrementale è un modello di sviluppo software che prevede la consegna di funzionalità in maniera incrementale, cioè il prodotto finale viene 
    sviluppato attraverso una serie di rilasci parziali}
}
\newglossaryentry{sprint}
{
    name=\glslink{sprint}{sprint},
    text=sprint,
    sort=sprint,
    description={Lo sprint è un periodo di tempo breve, della durata di una o due settimane, in cui viene sviluppata 
    una funzionalità del prodotto finale}
}
\newglossaryentry{Product Backlog}
{
    name=\glslink{Product Backlog}{Product Backlog},
    text=Product Backlog,
    sort=Product Backlog,
    description={Il Product Backlog è un elenco ordinato di requisiti che rappresentano le funzionalità del prodotto finale}
}
\newglossaryentry{agile}
{
    name=\glslink{agile}{agile},
    text=agile,
    sort=agile,
    description={Il termine agile si riferisce a un insieme di metodi di sviluppo software che si basano su un approccio iterativo e incrementale}
}
\makeglossaries

\bibliography{appendix/bibliography}

\defbibheading{bibliography} {
    \cleardoublepage
    \phantomsection
    \addcontentsline{toc}{chapter}{\bibname}
    \chapter*{\bibname\markboth{\bibname}{\bibname}}
}

% Spacing between entries
\setlength\bibitemsep{1.5\itemsep}

\DeclareBibliographyCategory{opere}
\DeclareBibliographyCategory{web}

\addtocategory{opere}{womak:lean-thinking}
\addtocategory{web}{site:agile-manifesto}

\defbibheading{opere}{\section*{Riferimenti bibliografici}}
\defbibheading{web}{\section*{Siti Web consultati}}


\captionsetup{
    tableposition=top,
    figureposition=bottom,
    font=small,
    format=hang,
    labelfont=bf
}

\hypersetup{
    %hyperfootnotes=false,
    %pdfpagelabels,
    colorlinks=true,
    linktocpage=true,
    pdfstartpage=1,
    pdfstartview=,
    breaklinks=true,
    pdfpagemode=UseNone,
    pageanchor=true,
    pdfpagemode=UseOutlines,
    plainpages=false,
    bookmarksnumbered,
    bookmarksopen=true,
    bookmarksopenlevel=1,
    hypertexnames=true,
    pdfhighlight=/O,
    %nesting=true,
    %frenchlinks,
    urlcolor=webbrown,
    linkcolor=RoyalBlue,
    citecolor=webgreen
    %pagecolor=RoyalBlue,
}

% Delete all links and show them in black
\if \isprintable 1
    \hypersetup{draft}
\fi

% Listings setup
\lstset{
    language=[LaTeX]Tex,%C++,
    keywordstyle=\color{RoyalBlue}, %\bfseries,
    basicstyle=\small\ttfamily,
    %identifierstyle=\color{NavyBlue},
    commentstyle=\color{Green}\ttfamily,
    stringstyle=\rmfamily,
    numbers=none, %left,%
    numberstyle=\scriptsize, %\tiny
    stepnumber=5,
    numbersep=8pt,
    showstringspaces=false,
    breaklines=true,
    frameround=ftff,
    frame=single
}

\definecolor{webgreen}{rgb}{0,.5,0}
\definecolor{webbrown}{rgb}{.6,0,0}

\newcommand{\sectionname}{sezione}
\addto\captionsitalian{\renewcommand{\figurename}{Figura}
                       \renewcommand{\tablename}{Tabella}}

\newcommand{\glsfirstoccur}{\ap{{[g]}}}

\newcommand{\intro}[1]{\emph{\textsf{#1}}}

% Risks environment
\newcounter{riskcounter}                % define a counter
\setcounter{riskcounter}{0}             % set the counter to some initial value

%%%% Parameters
% #1: Title
\newenvironment{risk}[1]{
    \refstepcounter{riskcounter}        % increment counter
    \par \noindent                      % start new paragraph
    \textbf{\arabic{riskcounter}. #1}   % display the title before the content of the environment is displayed
}{
    \par\medskip
}

\newcommand{\riskname}{Rischio}

\newcommand{\riskdescription}[1]{\textbf{\\Descrizione:} #1.}

\newcommand{\risksolution}[1]{\textbf{\\Soluzione:} #1.}

% Use case environment
\newcounter{usecasecounter}             % define a counter
\setcounter{usecasecounter}{0}          % set the counter to some initial value

%%%% Parameters
% #1: ID
% #2: Nome
\newenvironment{usecase}[2]{
    \renewcommand{\theusecasecounter}{\usecasename #1}  % this is where the display of
                                                        % the counter is overwritten/modified
    \refstepcounter{usecasecounter}             % increment counter
    \vspace{10pt}
    \par \noindent                              % start new paragraph
    {\large \textbf{\usecasename #1: #2}}       % display the title before the
                                                % content of the environment is displayed
    \medskip
}{
    \medskip
}

\newcommand{\usecasename}{UC}

\newcommand{\usecaseactors}[1]{\textbf{\\Attori Principali:} #1. \vspace{4pt}}
\newcommand{\usecasepre}[1]{\textbf{\\Precondizioni:} #1. \vspace{4pt}}
\newcommand{\usecasedesc}[1]{\textbf{\\Descrizione:} #1. \vspace{4pt}}
\newcommand{\usecasepost}[1]{\textbf{\\Postcondizioni:} #1. \vspace{4pt}}
\newcommand{\usecasealt}[1]{\textbf{\\Scenario Alternativo:} #1. \vspace{4pt}}

% Namespace description environment
\newenvironment{namespacedesc}{
    \vspace{10pt}
    \par \noindent  % start new paragraph
    \begin{description}
}{
    \end{description}
    \medskip
}

\newcommand{\classdesc}[2]{\item[\textbf{#1:}] #2}
\setcounter{tocdepth}{4}