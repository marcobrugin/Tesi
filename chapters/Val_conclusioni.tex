\chapter{Valutazioni e Conclusioni}
\label{cap:conclusioni}
\section{Raggiungimento degli obiettivi}
Per quanto riguarda gli obiettivi prefissati all'inizio del periodo di stage si può affermare che sono stati raggiunti tutti.\\
In particolare sono stati soddisfatti tutti gli obiettivi obbligatori e facoltativi. Di seguito la tabella riassuntiva \ref{tab:obiettivi_raggiunti}.\\


\begin{table}[h]
    \centering
    \caption{Tabella riassuntiva degli obiettivi raggiunti}
    \label{tab:obiettivi_raggiunti}
    \begin{tabular}{|c|c|c|}
        \hline
        \textbf{Obiettivo} & \textbf{Importanza} & \textbf{Stato} \\\hline
        \textbf{O1} & Obbligatorio & Soddisfatto \\\hline
        \textbf{O2} & Obbligatorio & Soddisfatto\\\hline
        \textbf{O3} & Obbligatorio & Soddisfatto\\\hline
        \textbf{O4} & Obbligatorio & Soddisfatto \\ \hline
        \textbf{O5} & Obbligatorio & Soddisfatto \\\hline
        \textbf{O6} & Obbligatorio & Soddisfatto \\\hline
        \textbf{O7} & Obbligatorio & Soddisfatto \\\hline
        \textbf{F1} & Facoltativo & Soddisfatto\\\hline
        \textbf{F2} & Facoltativo & Soddisfatto \\\hline
        \textbf{F3} & Facoltativo & Soddisfatto\\\hline
    \end{tabular} 
\end{table}

\section{Valutazione del lavoro svolto}
Nonostante la mia scarsa conoscenza dell'ambito applicativo e tecnologico in cui rientrava il progetto di stage,
il prodotto finale sviluppato è riuscito a soddisfare le aspettative del tutor aziendale.\\
Inoltre l'indagine svolta 
ha portato, sia a me stesso che all'azienda,
 molti benefici e conoscenze da impiegare in progetti futuri.\\
La documentazione presente nei vari siti web ufficiali è stata sufficiente per poter comprendere 
da un punto di vista teorico, ma insufficiente 
per poterli utilizzare in modo efficace dal punto di vista pratico. Ritengo che 
le attività di \gls{hands-on}{} svolte sono state di fondamentale importanza, in particolar modo 
quando svolte in collaborazione con altri colleghi, perchè mi hanno permesso di 
scoprire funzionalità e caratteristiche degli strumenti, che non sono presenti nella documentazione ufficiale.\\
\section{Divario rispetto al percorso di studi}
Nonostante, nel percorso di stage, abbia incontrato strumenti e tecnologie innovative, ritengo che il percorso di studi 
mi abbia fornito le basi necessarie per poter completare con successo lo stage.\\
Infatti, anche se gli insegnamenti del Corso di Laurea, non hanno trattato direttamente gli argomenti affrontati durante lo stage,
mi hanno comunque fornito le conoscenze di base per poter comprendere e approfondire gli argomenti trattati;
in un settore in continua evoluzione come quello dell'informatica, riuscire a comprendere e adattarsi a nuove tecnologie è fondamentale ed è 
stato molto personalmente molto apprezzato.
\section{Valutazione personale}
Nel complesso ritengo che l'esperienza di stage sia stata molto positiva, perchè 
mi ha permesso una crescita e maturazione personale. In particolare mi ha consentito di cimentarmi
migliorando e potenziando le mie capacità di lavorare in team.\\
In conclusione grazie all'esperienza di stage ho potuto conoscere e migliorare abilità di me stesso
 che mi permetteranno di affrontare con maggiore consapevolezza e sicurezza le sfide future.\\
\newpage
\pagestyle{empty}
\null % o \mbox{} o \phantom{X}
\newpage