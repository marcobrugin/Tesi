\chapter{Valutazioni e Conclusioni}
\label{cap:conclusioni}
\section{Raggiungimento degli obiettivi}
L'oggetto del tirocinio, illustrato nel seguente documento, è stato lo studio e l'implementazione di una \gls{Data Pipeline}{}, eseguita all'interno di un ambiente che fa uso di \gls{container}{}, in grado di 
soddisfare esigenze di scalabilità, alta efficienza e \gls{fault tolerance}{} nell'elaborazione di grandi moli di dati.\\
Per il raggiungimento di tale obiettivo, sono state utilizzati \textbf{Apache Kafka}, come strumento di raccolta dati e 
\textbf{Apache Druid}, come strumento di trasformazione e analisi dati.\\
A completamento dello studio teorico condotto su tali strumenti, 
dopo aver creato e configurato una \gls{Data Pipeline}{}, con le caratteristiche sopra descritte, 
sono stati eseguiti una serie di test, finalizzati a comparare 
le prestazioni di esecuzione di uno strumento \gls{olap}{}, come \textbf{Apache Druid}, con 
un classico database relazione, come \textbf{PostgreSQL} ed a evidenziare la facilità con cui \textbf{Druid} è in grado di 
fornire funzionalità onerose in altri sistemi di elaborazione dati.\\
Per quanto riguarda gli obiettivi prefissati all'inizio del periodo di stage, riportati nella sezione \ref{sec:obiettivi}, si può affermare che sono stati raggiunti tutti.\\
In particolare sono stati soddisfatti tutti gli obiettivi obbligatori e facoltativi. Di seguito la tabella riassuntiva \ref{tab:obiettivi_raggiunti}.\\
\begin{table}[H]
    \centering
    \caption{Tabella riassuntiva degli obiettivi raggiunti}
    \label{tab:obiettivi_raggiunti}
    \begin{tabular}{|c|c|c|}
        \hline
        \textbf{Obiettivo} & \textbf{Importanza} & \textbf{Stato} \\\hline
        \textbf{O1} & Obbligatorio & Soddisfatto \\\hline
        \textbf{O2} & Obbligatorio & Soddisfatto\\\hline
        \textbf{O3} & Obbligatorio & Soddisfatto\\\hline
        \textbf{O4} & Obbligatorio & Soddisfatto \\ \hline
        \textbf{O5} & Obbligatorio & Soddisfatto \\\hline
        \textbf{O6} & Obbligatorio & Soddisfatto \\\hline
        \textbf{O7} & Obbligatorio & Soddisfatto \\\hline
        \textbf{F1} & Facoltativo & Soddisfatto\\\hline
        \textbf{F2} & Facoltativo & Soddisfatto \\\hline
        \textbf{F3} & Facoltativo & Soddisfatto\\\hline
    \end{tabular} 
\end{table}
\pagebreak
\noindent
La realizzazione di un prototipo di una \gls{Data Pipeline}{} ha mostrato la fattibilità della creazione di un sistema contenente un flusso automatizzato
di dati, a partire da un sorgente di raccolta dati, come \textbf{Apache Kafka} e terminante in un sistema ad alta efficienza, portabile, scalabile e resiliente di analisi dati, come \textbf{Apache Druid}.
Tutto ciò, tramite uno studio teorico delle tecnologie in esame e l'implementazione di quanto citato pocanzi, ha permesso di 
acquisire conoscenze e competenze relative alle \textbf{architetture distribuite} e in ambito \gls{olap}{} e \gls{eda}{}.  
\section{Valutazione del lavoro svolto}
Nonostante la mia scarsa conoscenza dell'ambito applicativo e tecnologico in cui rientrava il progetto di stage,
il prodotto finale sviluppato è riuscito a soddisfare le aspettative del tutor aziendale.\\
Inoltre l'indagine svolta 
ha portato, sia a me stesso che all'azienda,
 molti benefici e conoscenze da impiegare in progetti futuri.\\
La documentazione presente nei vari siti web ufficiali è stata sufficiente per poter comprendere gli strumenti in esame
da un punto di vista teorico, ma insufficiente 
per poterli utilizzare in modo efficace dal punto di vista pratico. Ritengo che 
le attività di \gls{hands-on}{} siano state di fondamentale importanza, in particolar modo 
quando svolte in collaborazione con altri colleghi, perchè mi hanno permesso di 
scoprire funzionalità e caratteristiche degli strumenti, che sono descritte in modo molto limitato nella documentazione ufficiale.\\
Sebbene, nel percorso di stage, abbia incontrato strumenti e tecnologie innovative, ritengo che il percorso di studi 
mi abbia fornito le basi necessarie per poter completare con successo lo stage.\\
Infatti, anche se gli insegnamenti del Corso di Laurea, non hanno trattato direttamente gli argomenti affrontati durante lo stage,
mi hanno comunque fornito le conoscenze di base per poter comprendere e approfondire gli argomenti trattati.\\
Nel complesso ritengo che l'esperienza di stage sia stata molto positiva, perchè 
mi ha permesso una crescita e maturazione personale. In particolare mi ha consentito di
migliorare e potenziare le mie capacità di lavorare in team.\\
In conclusione grazie all'esperienza di stage ho potuto conoscere e migliorare abilità di me stesso
 che mi permetteranno di affrontare con maggiore consapevolezza e sicurezza le sfide future.\\
\newpage
\pagestyle{empty}
\null % o \mbox{} o \phantom{X}
\newpage