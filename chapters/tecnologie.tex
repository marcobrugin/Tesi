\chapter{Tecnologie e strumenti utilizzati}\label{cap:Tecnologie e strumenti utilizzati}
Per il raggiungimento degli obiettivi del progetto di stage sono state utilizzate diverse tecnologie e strumenti. 
In questa sezione verranno riepilogate con una breve descrizione del loro utilizzo. 
\section{Linguaggi utilizzati}
\subsection{YAML}
YAML, acronimo di YAML Ain't Markup Language, è un linguaggio di Markup, noto per la sua leggibilità e la sua 
chiarezza espressiva. \\
La prima idea attorno al linguaggio YAML nasce attorno agli anni '90 quando Clark C. Evans, software developer, lo propone come alternativa a XML.\\
Nel 2001 Evans pubblica la prima specifica del linguaggio, che va a definire i principi fondamentali del linguaggio.\\
Negli anni YAML ha acquisito sempre più popolarità e interesse di utilizzo, in quanto ha offerto una configurazione semplice e leggibile 
per strumenti si DEVOPS, orchestrazione, automazione e molto altro (Figura \ref{fig:yaml}).\\
La storia di YAML è strettamente legata alla esigenza di semplificare la rappresentazione di dati complessi, 
in un formato più comprensibile a un essere umano e a macchine.\\
\begin{figure}[hpp]
    \centering
    \includegraphics[width=0.5\textwidth]{images/tecnologie/logo_yaml.png}
    \caption{Logo di YAML}
    \label{fig:yaml}
\end{figure}
\pagebreak
\subsection{Python}
Python è un linguaggio di programmazione ad alto livello, orientato agli oggetti,
che si distingue per la sua sintassi chiara e intuitiva (Figura \ref{fig:python}).\\
Creato da Guido van Rossum e rilasciato per la prima volta nel 1991, è cresciuto fino a 
diventare uno dei linguaggi più utilizzati al mondo.\\
Data la sua semplicità e la sua versatilità, Python è utilizzato in diversi ambiti dallo sviluppo web, alla \gls{Data Analytics}{}, allo sviluppo di applicazione 
desktop e mobile, fino ad arrivare all'automazione e all'intelligenza artificiale.\\
\begin{figure}[hpp]
    \centering
    \includegraphics[width=0.5\textwidth]{images/tecnologie/logo_python.png}
    \caption{Logo di Python}
    \label{fig:python}
\end{figure}
\section{Tecnologie utilizzate}
\subsection{Apache Kafka}

\subsection{Apache Druid}


\subsection{Ambienti containerizzati}
\subsubsection{Docker Compose}
\subsection{Strumenti di gestione}
\subsubsection{Click Up}   %strumento di issue tracking system utilizzato
\section{Ambiente di lavoro}
\subsection{Metodologia di sviluppo}
\subsection{Ambiente di sviluppo}
\subsection{Versioning}
\subsection{Documentazione}
