\chapter{Il percorso di stage }\label{cap:Il_percorso}
\section{Formazione}
Il processo di formazione che mi è stato fornito ha avuto un ruolo fondamentale nella buona riuscita del progetto di stage, 
ha avuto una durata di circa 4 settimane. \\
La causa del protrarsi del processo di formazione è stata provocata dal fatto che concetti legati all'architettura \gls{eda}{},
\textbf{Apache Kafka}, \textbf{Apache Druid}, \textbf{Docker Compose} sono state del tutto innovative per me.\\
Tutto il processo di formazione è stato tracciato e monitorato dal tutor aziendale e da me stesso attraverso le \gls{board}{} offerte 
dal software di project management \textbf{ClickUp} (Figura \ref{cap:ClickUp}).\\
\begin{figure}[h]
    \centering
    \includegraphics[width=1\textwidth]{images/percorso/formazione.png}
    \caption{Board di ClickUp per il processo di formazione}
    \label{cap:ClickUp}
\end{figure}
\pagebreak
\\
Inoltre durante il processo di formazione, oltre a reperire informazioni da documentazione ufficiale fornita , ho avuto anche modo 
di approfondire quanto appena appreso attraverso delle attività di \gls{hands-on}{} che mi hanno permesso di mettere in pratica quanto appreso (Figura \ref{cap:Hands-on}).
\begin{figure}[h]
    \centering
    \includegraphics[width=1\textwidth]{images/percorso/hands_on.png}
    \caption{Attività di hands-on per il processo di formazione}
    \label{cap:Hands-on}
\end{figure}
\\
Oltre a ciò, durante il processo di formazione, in collaborazione con il tutor aziendale, è stato definito un processo di coordinamento e produzione di 
documentazione tecnica che mi ha permesso durante tutto lo svolgimento del percorso di stage di avere un tracciamento dei concetti appresi 
e di avere riferimenti per la risoluzione di problemi o dubbi sorti (Figura \ref{cap:Documentazione})
\begin{figure}[h]
    \centering
    \includegraphics[width=1\textwidth]{images/percorso/coordinamento.png}
    \caption{Board di ClickUp per il processo di coordinamento e documentazione}
    \label{cap:Documentazione}
\end{figure}
\subsection{Daily stand-up meeting}
Durante tutto il percorso di stage, in collaborazione con il tutor aziendale, è stato definito in concomitanza con l'inizio del processo di formazione
è stato definito un processo di \textbf{supporto} in stile \gls{Scrum}{}, andando fissare degli incontri giornalieri di circa 15 minuti finalizzati a:
\begin{list}{*}
    \item \textbf{Monitorare} lo stato di avanzamento delle attività svolte, da svolgere e in corso di svolgimento;
    \item \textbf{Risolvere} eventuali dubbi o problemi sorti durante lo svolgimento delle attività;
    \item \textbf{Definire} eventuali miglioramenti o cambiamenti da apportare alle da svolgere o in corso di svolgimento.
\end{list}
\section{Configurazione}
\subsection{Configurazione di un cluster Kafka con Docker Compose}
Dopo aver terminato le attività di formazione su \textbf{Apache Kafka} e \textbf{Docker Compose} 
ho iniziato la configurazione di un cluster Kafka con Docker Compose. 
\\Seguendo la buona pratica dell'alta affidabilità, descritta del 
paragrafo \ref{sec:alta_affidabilita}, ho configurato un cluster \textbf{Kafka} con 3 nodi, 1 nodo \textbf{Zookeeper}.\\
Per la configurazione ho utilizzato il seguente \textbf{docker-compose.yml} file, di cui ne viene riportato solo la parte relativa alla configurazione di un nodo \textbf{Kafka} gli altri nodi sono configurati in maniera analoga.
\begin{lstlisting}
  networks:
    kafka-druid:
      name: kafka-druid
      driver: bridge
      external: true
  
  services:
    zookeeper:
      container_name: zookeeper
      hostname: zookeeper
      image: confluentinc/cp-zookeeper:7.4.0
      networks: 
        - kafka-druid
      ports:
        - "2181:2181"
      environment:
        - ZOOKEEPER_SERVER_ID=1
        - ZOOKEEPER_CLIENT_PORT=2181
    kafka:
      image: confluentinc/cp-kafka:7.4.0
      hostname: kafka
      container_name: kafka
      networks:
        - kafka-druid
      ports:
        - "29092:29092"
      environment:
        KAFKA_ADVERTISED_LISTENERS: INTERNAL://kafka:9092,EXTERNAL://localhost:29092
        KAFKA_LISTENER_SECURITY_PROTOCOL_MAP: INTERNAL:PLAINTEXT,EXTERNAL:PLAINTEXT
        KAFKA_INTER_BROKER_LISTENER_NAME: INTERNAL
        KAFKA_ZOOKEEPER_CONNECT: "zookeeper:2181"
        KAFKA_BROKER_ID: 1
        KAFKA_LOG4J_LOGGERS: "kafka.controller=INFO,kafka.producer.async.DefaultEventHandler=INFO,state.change.logger=INFO"
        KAFKA_OFFSETS_TOPIC_REPLICATION_FACTOR: 1
        KAFKA_TRANSACTION_STATE_LOG_REPLICATION_FACTOR: 1
        KAFKA_TRANSACTION_STATE_LOG_MIN_ISR: 1
        KAFKA_AUTHORIZER_CLASS_NAME: kafka.security.authorizer.AclAuthorizer
        KAFKA_ALLOW_EVERYONE_IF_NO_ACL_FOUND: "true"  
\end{lstlisting}    



\subsection{Configurazione di un cluster Druid con Docker Compose}
\subsubsection{Configurazione dell'enviroment di Apache Druid}

\section{Esecuzione e testing}
\newpage
\pagestyle{empty}
\null % o \mbox{} o \phantom{X}
\newpage