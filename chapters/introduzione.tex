\chapter{Introduzione}
\label{cap:introduzione}
\section{Descrizione dell'azienda}
Sync Lab s.r.L. è una azienda italiana attiva nell'abito \gls{ict}{}, specializzata nello sviluppo e consulenza IT dal 2002 con sedi a 
Milano, Roma, Napoli, Verona e Como. È una azienda orientata verso la Business Innovation, finalizzata alla 
creazione di soluzioni innovative che abbracciano i nuovi paradigmi della trasformazione digitale.
Sync Lab possiede numerose certificazioni ISO LL-C per l'attestazione della 
qualità dei prodotti e servizi offerti. In particolare possiede le certificazioni 
ISO-9001 per la qualità dell'azienda, ISO-14001 per l'adozione quadro sistematico per l'integrazione delle pratiche a protezione dell'ambiente, ISO-27001 per la definizione di un \gls{sgsi}{}, ISO-45001 per l'adozione di un \gls{ohsg}{}.
\\
Attualmente Sync Lab dal punto di vista dell'organico è composta da più 300 dipendenti e lavora per più di 150 clienti diretti e finali, tra i più rilevanti ci sono nomi come: TIM, Trenitalia, PosteItaliane, UniCredit, ENI, ENEL, Vodafone, Fastweb.
\\
Sync Lab è un'azienda che si pone come obiettivo quello di essere un punto di riferimento per i propri clienti nella realizzazione di prodotti e soluzioni innovative per diversi settori di mercato, tra i quali: Sanità, Industria, Energia, Telco, Finanza e Trasporti \& Logistica.
\begin{figure}[htbp]  
\centering
    \includegraphics[width=0.5\textwidth]{images/introduzione/logo_azienda.png}
    \caption{Logo dell'aziedan Sync Lab s.r.L.}
\end{figure}
\pagebreak
\section{Idea di fondo del progetto}
Oggigiorno la gestione e l'analisi di grandi moli di dati in tempo reale sta diventando fondamentale 
per le aziende che vogliono rimanere competitive sul mercato. \\ 
Per questo motivo è necessario utilizzare tecnologie e software che permettano di analizzare e archiviare 
i dati in modo efficiente e veloce. \\
D'altro canto, però, è necessario anche che queste tecnologie siano in grado di scalare in modo verticale e orizzontale in base al carico 
di lavoro da sostenere. Inoltre è necessario che queste tecnologie siano in grado di garantire un elevato livello affidabilità. \\
Per questo motivo Sync Lab ha deciso d' investire in un progetto di ricerca e sviluppo che ha come obiettivo quello di creare 
una \gls{Data Pipeline}{} in grado di garantire le caratteristiche sopra descritte. \\
L'azienda ha già a disposizione un sistema di raccolta dati in  real time, basato su Apache Kafka, che permette di ricevere dati da
diversi sistemi e applicazioni, ma vuole andarlo a integrare con un sistema di analisi in real time permetta di eseguire operazioni 
sui dati ricevuti prima di archiviarli. \\
Particolarmente importante dovrà essere la fase di analisi dei dati, in quanto dovrà essere possibile eseguire operazioni di aggregazione
per aumentare la successiva estrazione dei dati. 
\subsection{Il ruolo dello stagista}
Lo stagista ha un ruolo fondamentale in tale tipologia di progetto, infatti sarà colui a portare uno spirito d'innovazione e consolidare il valore aggiunto 
aziendale. \\
Il percorso che lo stagista dovrà intraprendere sono state elencate all'interno di un \textit{Piano di lavoro}, che aiuterà il tutor 
aziendale designato da Sync Lab a guidare e valutare il lavoro svolto dallo stagista. \\
Inoltre al termine del periodo di stage, sotto la supervisione del tutor aziendale è stata pianificata una presentazione rivolta a tutti \textit{stakeholder} aziendali, mirata a 
mostrare i risultati ottenuti con le tecnologie utilizzate e mettere in risalto le potenzialità del prototipo sviluppato. D'altra parte tale presentazione sarà anche l'occasione 
per un confronto per far emergere eventuali criticità o difetti,  e miglioramenti da apportare al prototipo sviluppato.
\pagebreak
\section{Il progetto di stage}
\subsection{Descrizione del progetto}
Le attività descritte nel seguente in questa tesi si basano sulla progettazione e realizzazione di un prototipo di \gls{Data Pipeline}{} in grado di ricevere dati in real time da un sistema di raccolta dati basato su Apache Kafka, eseguire operazioni di aggregazione con Apache Druid in modo efficiente per facilitarne la successiva estrazione, fornendo così in output dati proonti per l'analisi.\\
In generale una \gls{Data Pipeline}{} è costituita da tre elementi sostanziali: una origine, una o più fasi di trasformazione e una o più destinazioni.
Inoltre per facilitare le operazioni di deploy è richiesto che il prototipo sia eseguibile con Docker Compose. \\
Il progetto in sè fa parte di una rivoluzione tecnologica che Sync Lab sta portando avanti nel campo Data Processing e Data Analytics. \\
In particolare il progetto di stage si pone come obiettivo quello di andare ad analizzare le prestazioni di Apache Druid rispetto ad un classico database relazionale, come PostgreSQL, in termini di velocità di esecuzione delle query e di scalabilità in relazione alle funzionalità di aggregazione fornite da tale strumento. 
\subsection{Obiettivi formativi}
In generale lo stage ha come obiettivo quello di far acquisire allo stagista concetti fondamentali riguardanti: 
\begin{itemize}
    \item Container technology;
    \item Apache Kafka e le Event Driven Architecture, design publish/subscribe;
    \item Column Based Database e la relazione/confronto tra questi e i classici DB relazionali SQL e quelli
    Documentali;
    \item Middleware, \gls{Data Pipeline}{}, le architetture distribuite, scalabili e resilienti.
\end{itemize}
\subsection{Risultati attesi e Obiettivi fissati}
Sono riportati nella Tabella 1.1 i risultati attesi e gli obiettivi fissati per lo stage con rispettivo identificativo, importanza e breve descrizione.
L'identificativo (riportato in breve con “ID”) è la sigla che identifica ogni requisito e rispetta la seguente notazione [Importanza][Identificativo]. \\
L’importanza è indicata dalla sigla O oppure F ad indicare rispettivamente un obiettivo
obbligatorio oppure facoltativo; mentre l’identificativo è un numero incrementale che
segnala in modo univoco l’obiettivo o il risultato in esame.\\
\\
Un obiettivo, o risultato viene classificato secondo l'importanza in: 
\begin{list}{*}{}
    \item \textbf{Obbligatorio (O)}: obiettivo irrinunciabile, vincolante in quanto necessario per l'avanzamento del progetto;
    \item \textbf{Facoltativo (F)}: obiettivo utile ma non essenziale, che può essere soddisfatto solo se tutti gli obiettivi obbligatori sono stati raggiunti e il cui soddisfatto rende il progetto più completo.     
\end{list} 
\pagebreak
\subsection{Pianificazione}

\subsection{Analisi preventiva dei rischi}


