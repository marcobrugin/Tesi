\chapter{Introduzione}
\label{cap:introduzione}
\section{Descrizione dell'azienda}
Sync Lab s.r.L. è una azienda italiana attiva nell'abito \gls{ict}{}, specializzata nello sviluppo e consulenza IT dal 2002 con sedi a 
Milano, Roma, Napoli, Verona e Como. È una azienda orientata verso la Business Innovation, finalizzata alla 
creazione di soluzioni innovative che abbracciano i nuovi paradigmi della trasformazione digitale.
Sync Lab possiede numerose certificazioni ISO LL-C per l'attestazione della 
qualità dei prodotti e servizi offerti. In particolare possiede le certificazioni 
ISO-9001 per la qualità dell'azienda, ISO-14001 per l'adozione quadro sistematico per l'integrazione delle pratiche a protezione dell'ambiente, ISO-27001 per la definizione di un \gls{sgsi}{}, ISO-45001 per l'adozione di un \gls{ohsg}{}.
\\
Attualmente Sync Lab dal punto di vista dell'organico è composta da più 300 dipendenti e lavora per più di 150 clienti diretti e finali, tra i più rilevanti ci sono nomi come: TIM, Trenitalia, PosteItaliane, UniCredit, ENI, ENEL, Vodafone, Fastweb.
\\
Sync Lab è un'azienda che si pone come obiettivo quello di essere un punto di riferimento per i propri clienti nella realizzazione di prodotti e soluzioni innovative per diversi settori di mercato, tra i quali: Sanità, Industria, Energia, Telco, Finanza e Trasporti \& Logistica.
\begin{figure}[htbp]  
\centering
    \includegraphics[width=0.5\textwidth]{images/introduzione/logo_azienda.png}
    \caption{Logo dell'aziedan Sync Lab s.r.L.}
\end{figure}
\pagebreak
\section{Idea di fondo del progetto}
Oggigiorno la gestione e l'analisi di grandi moli di dati in tempo reale sta diventando fondamentale 
per le aziende che vogliono rimanere competitive sul mercato. \\ 
Per questo motivo è necessario utilizzare tecnologie e software che permettano di analizzare e archiviare 
i dati in modo efficiente e veloce. \\
D'altro canto, però, è necessario anche che queste tecnologie siano in grado di scalare in modo verticale e orizzontale in base al carico 
di lavoro da sostenere. Inoltre è necessario che queste tecnologie siano in grado di garantire un elevato livello affidabilità. \\
Per questo motivo Sync Lab ha deciso d' investire in un progetto di ricerca e sviluppo che ha come obiettivo quello di creare 
una \gls{Data Pipeline}{} in grado di garantire le caratteristiche sopra descritte. \\
L'azienda ha già a disposizione un sistema di raccolta dati in  real time, basato su Apache Kafka, che permette di ricevere dati da
diversi sistemi e applicazioni, ma vuole andarlo a integrare con un sistema di analisi in real time permetta di eseguire operazioni 
sui dati ricevuti prima di archiviarli. \\
Particolarmente importante dovrà essere la fase di analisi dei dati, in quanto dovrà essere possibile eseguire operazioni di aggregazione
per aumentare la successiva estrazione dei dati. 
\subsection{Il ruolo dello stagista}
Lo stagista ha un ruolo fondamentale in tale tipologia di progetto, infatti sarà colui a portare uno spirito d'innovazione e consolidare il valore aggiunto 
aziendale. \\
Il percorso che lo stagista dovrà intraprendere sono state elencate all'interno di un \textit{Piano di lavoro}, che aiuterà il tutor 
aziendale designato da Sync Lab a guidare e valutare il lavoro svolto dallo stagista. \\
Inoltre al termine del periodo di stage, sotto la supervisione del tutor aziendale è stata pianificata una presentazione rivolta a tutti \textit{stakeholder} aziendali, mirata a 
mostrare i risultati ottenuti con le tecnologie utilizzate e mettere in risalto le potenzialità del prototipo sviluppato. D'altra parte tale presentazione sarà anche l'occasione 
per un confronto per far emergere eventuali criticità o difetti,  e miglioramenti da apportare al prototipo sviluppato.
\pagebreak
\section{Il progetto di stage}
\subsection{Descrizione del progetto}
Le attività descritte nel seguente in questa tesi si basano sulla progettazione e realizzazione di un prototipo di \gls{Data Pipeline}{} in grado di ricevere dati in real time da un sistema di raccolta dati basato su Apache Kafka, eseguire operazioni di aggregazione con Apache Druid in modo efficiente per facilitarne la successiva estrazione, fornendo così in output dati proonti per l'analisi.\\
In generale una \gls{Data Pipeline}{} è costituita da tre elementi sostanziali: una origine, una o più fasi di trasformazione e una o più destinazioni.
Inoltre per facilitare le operazioni di deploy è richiesto che il prototipo sia eseguibile con Docker Compose. \\
Il progetto in sè fa parte di una rivoluzione tecnologica che Sync Lab sta portando avanti nel campo Data Processing e Data Analytics. \\
In particolare il progetto di stage si pone come obiettivo quello di andare ad analizzare le prestazioni di Apache Druid rispetto ad un classico database relazionale, come PostgreSQL, in termini di velocità di esecuzione delle query e di scalabilità in relazione alle funzionalità di aggregazione fornite da tale strumento. 
\subsection{Obiettivi formativi}
In generale lo stage ha come obiettivo quello di far acquisire allo stagista concetti fondamentali riguardanti: 
\begin{itemize}
    \item Container technology;
    \item Apache Kafka e le Event Driven Architecture, design publish/subscribe;
    \item Column Based Database e la relazione/confronto tra questi e i classici DB relazionali SQL e quelli
    Documentali;
    \item Middleware, \gls{Data Pipeline}{}, le architetture distribuite, scalabili e resilienti.
\end{itemize}
\subsection{Risultati attesi e Obiettivi fissati}
Sono riportati nella Tabella \ref{tab:Tabella1} i risultati attesi e gli obiettivi fissati per lo stage con rispettivo identificativo, importanza e breve descrizione.
L'identificativo (riportato in breve con “ID”) è la sigla che identifica ogni requisito e rispetta la seguente notazione [Importanza][Identificativo]. \\
L’importanza è indicata dalla sigla O oppure F ad indicare rispettivamente un obiettivo
obbligatorio oppure facoltativo; mentre l’identificativo è un numero incrementale che
segnala in modo univoco l’obiettivo o il risultato in esame.\\
\\
Un obiettivo, o risultato viene classificato secondo l'importanza in: 
\begin{list}{*}{}
    \item \textbf{Obbligatorio (O)}: obiettivo irrinunciabile, vincolante in quanto necessario per l'avanzamento del progetto;
    \item \textbf{Facoltativo (F)}: obiettivo utile ma non essenziale, che può essere soddisfatto solo se tutti gli obiettivi obbligatori sono stati raggiunti e il cui soddisfatto rende il progetto più completo.     
\end{list} 

 \begin{table}[htbp]
    \centering
    \caption{Tabella degli obiettivi}    
    \label{tab:Tabella1}
    \begin{tabularx}{\textwidth}{|c|c|X|}
        \hline
        \textbf{ID} & \textbf{Importanza} & \textbf{Descrizione} \\\hline
        O1 & Obbligatorio & comprensione e definizione di una piccola \gls{Data Pipeline}{}  che  preveda il trattamento dei dati
        tramite Apache Kafka e Apache Druid \\\hline
        O2 & Obbligatorio & comprensione dei vantaggi e degli overhead  che le Event Driven Architecture portano con
        sé\\\hline
        O3 & Obbligatorio & comprensione del pattern publisher/subscriber \\\hline
        O4 & Obbligatorio & set-up di un cluster Apache Kafka in ambiente  containerizzato \\\hline
        O5 & Obbligatorio & gestione delle Time Series e dei Column-based Databases \\\hline
        O6 & Obbligatorio & comprensione delle differenze tra i database relazionali  classici e i Column-based Data-
        bases\\\hline
        O7 & Obbligatorio &comprensione dell’impiego e utilità dei middleware \\\hline
        F1 & Facoltativo & produrre documentazione e un pacchetto di configurazione  dell’ambiente di sviluppo e
        esecuzione  della \gls{Data Pipeline}{}\\\hline
        F2 & Facoltativo & produrre documentazione che riporti  le differenze  di performance  tra Apache Druid e altri
         database relazionali classici per alcune  operazioni OLAP \\\hline
        F3 & Facoltativo & realizzare una presentazione che illustri l’architettura  sviluppata  a personale di settore o
        Stakeholder \\\hline
    \end{tabularx} 

\end{table}

\pagebreak
\subsection{Pianificazione}
In accordo con l'azienda Sync Lab, la pianificazione delle attività da svolgere è stata suddivisa in periodi della durata di una settimana. \\
Nella Tabella2 sono riportate in dettaglio i periodi previsti per ogni settimana, specificando il numero di ore preventivate per ognuno e il periodo in cui ne è pianificato lo svolgimento.
\\
\\
\\
\textbf{Prima Settimana - Ambientamento e introduzione agli strumenti del caso di studio}\\
Durante questa prima settimana del percorso di stage è prevista un'attività di ambientamento e introduzione agli strumenti del caso di studio. \\
In particolare, in questa fase iniziale, lo stagista si occuperà di:
\begin{list}{*}{}
    \item Introduzione al contesto e agli strumenti utilizzati, \gls{Data Pipeline}{}, EDA, middleware, cause e
    scopi dei progetti di questo tipo;
    \item introduzione ad Apache Kafka; sperimentazione personale delle tecnologie coinvolte su Docker
    Compose.
\end{list} 
\pagebreak
\textbf{Seconda Settimana - Studio e analisi di Apache Kafka}\\
Nella seconda settimana si stage è prevista un'attività di studio, analisi e sperimentazione con Apache Kafka. \\
In particolare lo stagista si occuperà di:
\begin{list}{*}{}
    \item Installazione e configurazione di Apache Kafka in un prototipo di ambiente distribuito;
    \item sperimentazione di alcune modalità d'inserimento dati in Apache Kafka;
    \item set-up della High Availability ed Exactly Once delivery del cluster Apache Kafka.
\end{list}
\textbf{Terza Settimana - Studio e analisi di Apache Druid e delle differenze rispetto ai database
documentali e relazionali classici}\\
Nel corso della terza settimana di stage, lo stagista sarà impegnato in un'attività di studio, analisi e sperimentazione con Apache Druid e le differenze rispetto ai database documentali e relazionali classici. \\

In particolare lo stagista si occuperà di:
\begin{list}{*}{}
    \item intraprendere uno studio di Apache Druid, e dei database per la gestione delle Time Series e ai Column-based
    Databases;
    \item comprendere gli obiettivi che si prepongono gli strumenti e degli Use Case ad essi associati;
    \item individuare le differenze rispetto ai documentali e ai relazionali classici SQL.
\end{list}

\noindent \textbf{Quarta Settimana - Analisi e sperimentazione di Apache Druid}\\
La quarta settimana di stage è dedicata ad una attività di analisi e sperimentazione delle funzionalità di Apache Druid. \\
In particolare lo stagista si occuperà di:
\begin{list}{*}{}
    \item Installazione e sperimentazione di un cluster in ambiente containerizzato di Apache Druid, con Docker Compose;
    \item Sperimentazioni con esecuzioni di query e operazioni sui dati;
\end{list}

\noindent  \textbf{Quinta Settimana - Implementazione Data Pipeline nella sua interezza lato sorgente}\\
La quinta settimana lo stagista sarà impegnato  nella implementazione di una \gls{Data Pipeline}{}, in particolare si tratterà 
dell'inserimento dei in Apache Kafka. \\

\noindent  \textbf{Sesta Settimana - Implementazione Data Pipeline nella sua interezza lato destinazione}\\
La sesta settimana lo stagista sarà impegnato nella implementazione dell'integrazione tra Apache Kafka e Apache Druid. \\
Inoltre lo stagista sarà anche impegnato nella creazione di un generatore di dati per testare l'architettura sviluppata. \\
\pagebreak

\noindent \textbf{Settima Settimana - Analisi della configurazione di Apache Druid}\\
La settima settimana sarà dedicata analisi e configurazione di Apache Druid. \\
In paraticolare lo stagista si occuperà di:
\begin{list}{*}{}
    \item configurazione approfondita di Apache Druid: enviroment, segments, memory use;
    \item Confronto della performance tra Apache Druid e altri database relazionali classici per alcune operazioni
    OLAP.
\end{list}
\textbf{Ottava Settimana - Conclusione del progetto e documentazione}\\
L'ottava settimana sarà dedicata alla chiusura del progetto, al consolidamento delle conoscenze acquisite e alla preparazione 
di una presentazione del progetto svolto ad eventuali Stakeholder. \\
\\
\\
I dettagli delle tempistiche e delle attività pianificate sono riportate nella tabella \ref{tab:Tabella3}.\\\\
\begin{table}[h]
    \captionof{table}{Tabella delle attività pianificate}
    \label{tab:Tabella3}
\begin{tabularx}{\textwidth}{|c|c|X|}
	\hline
	\textbf{Periodo} & \textbf{Durata in ore} & \textbf{Descrizione dell'attività} \\\hline
	19/06 - 23/06 & \textbf{40} & Ambientamento e introduzione agli strumenti del caso di studio \\ \hline
    26/06 - 30/06 & \textbf{40} & Studio e analisi di Apache Kafka \\ \hline
    03/07 - 07/07 & \textbf{40} & Studio e analisi di Apache Druid e delle differenze rispetto ai database documentali e relazionali classici  \\ \hline
    10/07 - 14/07 & \textbf{40} & Analisi e sperimentazione di Apache Druid \\ \hline
    17/07 - 21/07 & \textbf{40} & Implementazione \gls{Data Pipeline}{} nella sua interezza lato sorgente \\ \hline
    24/07 - 28/07 & \textbf{40} & Implementazione \gls{Data Pipeline}{} nella sua interezza lato destinazione \\ \hline
    31/07 - 04/08 & \textbf{40} & Analisi della configurazione di Apache Druid \\ \hline
	07/08 - 11/08 & \textbf{40} & Conclusione del progetto e documentazione \\\hline
	\textbf{Totale ore} & \multicolumn{2}{|c|}{\textbf{320}} \\\hline
\end{tabularx}
\end{table}
\pagebreak
\subsection{Analisi preventiva dei rischi}
Durante la fase di analisi iniziale del progetto di stage, sono stati individuati i seguenti rischi, cui si è cercato di porre rimedio con le azioni di mitigazione indicate. \\
\begin{enumerate}
    \item \textbf{Inesperienza tecnologica}: il progetto prevede l'utilizzo di tecnologie con cui lo stagista non ha mai avuto a che fare. \\
    \textbf{Rischio}: Medio .\\
    \textbf{Soluzione}: Per mitigare tale rischio, è stato previsto un periodo di ambientamento e delle tecnologie coinvolte, in modo da poter affrontare il progetto con maggiore consapevolezza.
    \item \textbf{Scelte errate nella progettazione dell'architettura}: il progetto prevede la progettazione di un'architettura complessa, con molte componenti interagenti tra loro. \\
    \textbf{Rischio}: Alto .\\
    \textbf{Soluzione}: Per mitigare tale rischio, è stato previsto un periodo di analisi e progettazione dell'architettura, con il supporto del tutor aziendale, in modo da poter ovviare tale rischio.

\end{enumerate}    
\newpage
\pagestyle{empty}
\null % o \mbox{} o \phantom{X}

\newpage