\chapter{Introduzione}
\label{cap:introduzione}
\section{Contesto aziendale}
\subsection{Descrizione dell'azienda}
Sync Lab s.r.L. è una azienda italiana attiva nell'abito \gls{ict}{ICT}, specializzata nello sviluppo e consulenza IT dal 2002 con sedi a 
Milano, Roma, Napoli, Verona e Como. È una azienda orientata verso la Business Innovation, finalizzata alla 
creazione di soluzioni innovative che abbracciano i nuovi paradigmi della trasformazione digitale.
Sync Lab possiede numerose certificazioni ISO LL-C per l'attestazione della 
qualità dei prodotti e servizi offerti. In particolare possiede le certificazioni 
ISO-9001 per la qualità dell'azienda, ISO-14001 per l'adozione quadro sistematico per l'integrazione delle pratiche a protezione dell'ambiente, ISO-27001 per la definizione di un SGSI, ISO-45001 per l'adozione di un OH\&S.
\\
Attualmente Sync Lab dal punto di vista dell'organico è composta da più 300 dipendenti e lavora per più di 150 clienti diretti e finali, tra i più rilevanti ci sono nomi come: TIM, Trenitalia, PosteItaliane, UniCredit, ENI, ENEL, Vodafone, Fastweb.
Sync Lab è un'azienda che si pone come obiettivo quello di essere un punto di riferimento per i propri clienti nella realizzazione di prodotti e soluzioni innovative per diversi settori di mercato, tra i quali: Sanità, Industria, Energia, Telco, Finanza e Trasporti & Logistica.
    \centering
    \includegraphics[width=0.5\textwidth]{images/introduzione/logo_azienda.png}
    \caption{Logo dell'aziedan Sync Lab s.r.l.}
\end{figure}

\subsection{Ambito di business}


\section{Idea di fondo del progetto}

Introduzione all'idea dello stage.
\subsection{Il ruolo dello stagista}
\section{Il progetto di stage}
\subsection{Descrizione del progetto}
\subsection{Obiettivi formativi}
\subsection{Vincoli}
\subsection{Risultati attesi}
\subsection{Pianificazione}
\subsection{Rischi}

