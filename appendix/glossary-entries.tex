% Acronyms
\newacronym[description={\glslink{ohsg}{Occupational Health and Safety}}]
    {ohs}{OH\&S}{Occupational Health and Safety}
\newacronym[description={\glslink{sgsig}{Sistema Gestione Sicurezza Informazioni}}]
    {sgsi}{SGSI}{Sistema di Gestione della Sicurezza delle Informazioni}
\newacronym[description={\glslink{ictg}{Information and Communication Technology}}]
    {ict}{ICT}{Information and Communication Technology}
\newacronym[description={\glslink{olapg}{Online Analytical Processing}}]
    {olap}{OLAP}{Online Analytical Processing}
\newacronym[description={\glslink{tcpg}{Transmission Control Protocol}}]
    {tcp}{TCP}{Transmission Control Protocol}
\newacronym[description={\glslink{apig}{Application Programming Interface}}]
    {api}{API}{Application Programming Interface}
\newacronym[description={{Event Driven Architecture}}]
    {eda}{EDA}{Event Driven Architecture}
\newacronym[description={\glslink{jsong}{JavaScript Object Notation}}]
    {json}{JSON}{JavaScript Object Notation}
\newacronym[description={\glslink{gplg}{General Public License}}]
    {gpl}{GPL}{General Public License}
\newglossaryentry{gplg} {
    name=\glslink{gpl}{GPL},
    text=General Public License,
    sort=GPL,
    description={ in informatica con il termine di licenza \textit{Gnu Public License GNU} si intende  
    una licenza di software libera e open-source, sviluppata dalla \textit{Free Software Foundation}, finalizzata a stabilire i termini e le condizioni 
    di utilizzo, copia, modifica e ridistribuzione del software o di un suo derivato}
}
\newglossaryentry{jsong}
{
    name={JSON},
    text=JavaScript Object Notation,
    sort=JSON,
    description={è l'acronimo di \textit{JavaScript Object Notation}, un formato di scrittura leggero basato su testo per rappresentare dati strutturati. 
    È comunemente utilizzato per trasmettere dati indipendentemente dal linguaggio di programmazione utilizzato. \textbf{JSON} presenta una sintassi semplice e comprensibile anche agli 
    esseri umani, attraverso la quale i dati vengono rappresentati in coppie chiave-valore, con la possibilità di poter includere 
    oggetti, array, numeri, stringhe, booleani e valori null}
}
\newglossaryentry{apig}
{
    name={API},
    text=Application Programming Interface,
    sort=API,
    description={ in informatica con il termine di \textit{Application Programming Interface API} si intende un insieme di interfacce di 
    programmazione, regole, protocolli e strumenti che consentono a diverse applicazioni di comunicare tra loro. La finalità delle 
    \textbf{API} è quella di definire un'astrazione tra hardware e programmatore o software a basso ed alto livello, in modo da facilitare
    la programmazione e l'interazione tra i diversi componenti di un sistema}
}
\newglossaryentry{tcpg} {
    name=\glslink{tcp}{TCP},
    text=Transmission Control Protocol,
    sort=TCP,
    description={è l'acronimo di \textit{Transmission Control Protocol}, uno dei principali protocolli di rete del \textit{TCP/IP}. Si tratta di un 
    protocollo al livello di trasporto, affidabile e orientato alla connessione. Viene ampiamente utilizzato per applicazioni 
    che richiedono comunicazioni affidabili e corenti}
}
\newglossaryentry{olapg} {
    name=\glslink{olap}{OLAP},
    text=Online Analytical Processing,
    sort=olap,
    description={è l'acronimo di \textit{Online Analytical Processing}, un insieme di tecniche e strumenti che permettono di analizzare i dati in modo 
    interattivo da più prospettive. Lo scopo principale di tali tecniche è quello di fornire, attraverso aggregazioni e compattazioni dei dati, informazioni 
    significative e approfondite finalizzate a supportare i processi decisiionali
    }
}
\newglossaryentry{ictg} {
    name=\glslink{ict}{ICT},
    text=Information and Communication Technology,
    sort=ict,
    description={in informatica, è l'acronimo di \textit{Information and Communication Technology}, un'ampia gamma di tecnologie che riguardano la raccolta, 
    elaborazione, trasmissione e condivisione di dati e informazioni}
}
\newglossaryentry{sgsig} {
    name=\glslink{sgsi}{SGSI},
    text=Sistema Gestione Sicurezza Informazioni,
    sort=sgsi,
    description={in informatica, è l'acronimo di \textit{Sistema di Gestione della Sicurezza delle Informazioni}, un insieme coordinato di politiche, procedure, strutture di un'organizzazione finalizzate a 
    gestire e proteggere informazioni sensibili e dati critici. Lo scopo principale di tale sistema è garantire la riservatezza, l'integrità 
    e la disponibilità delle informazioni aziendali}
}
\newglossaryentry{ohsg} {
    name=\glslink{ohs}{OH\&S},
    text=Occupational Health and Safety,
    sort=OH\&S,
    description={è un sistema di gestione che permette di gestire in modo strutturato la salute e sicurezza dei lavoratori}
}
\newglossaryentry{Data Pipeline}{
    name={Data Pipeline},
    text=Data Pipeline,
    sort=Data Pipeline,
    description={è un insieme di operazioni che permettono di trasformare e analizzare i dati in modo da renderli pronti per l'archiviazione}
}
\newglossaryentry{Data Processing}
{
    name={Data Processing},
    text=Data Processing,
    sort=Data Processing,
    description={è un sistema di manipolazione, trasformazione e analisi di dati grezzi al fine di ottenere informazioni significative e approfondite. Comprende una serie di passaggi che permettono di convertire dati non strutturati in forme più utili e che facilitano la loro elaborazione}
}
\newglossaryentry{Data Analytics}
{
    name={Data Analytics},
    text=Data Analytics,
    sort=Data Analytics,
    description={è il processo che permette di esaminare i dati per trarne conclusioni sull'informazione che contengono}
}
\newglossaryentry{Business Innovation}
{
    name={Business Innovation},
    text=Business Innovation,
    sort=Business Innovation,
    description={è un processo che permette d' introdurre nuovi metodi, idee, prodotti e servizi per migliorare l'efficienza, la produttività e la competitività di un'organizzazione}
}
\newglossaryentry{Middleware}
{
    name={Middleware},
    text=Middleware,
    sort=Middleware,
    description={è un software che si interpone tra un sistema operativo e le applicazioni che vengono eseguite al di sopra. Il suo scopo è quello di facilitare lo sviluppo di applicazioni e di nascondere la complessità del sistema operativo sottostante}
}
\newglossaryentry{repository}
{
    name={repository},
    text=repository,
    sort=repository,
    description={è un ambiente di archiviazione centralizzato in cui vengono conservati e gestiti i dati}
}
\newglossaryentry{metadati}
{
    name={metadati},
    text=metadati,
    sort=metadati,
    description={sono informazioni finalizzati a descrivere altre strutture dati, in modo da renderle più comprensibili e facili da individuare, analizzare e utilizzare}
}
\newglossaryentry{working directory}
{
    name={working directory},
    text=working directory,
    sort=working directory,
    description={è la directory di lavoro corrente}
}
\newglossaryentry{Scrum}
{
    name={Scrum},
    text=Scrum,
    sort=Scrum,
    description={è un framework agile per la gestione del ciclo di sviluppo del software}
}
\newglossaryentry{modello incrementale}
{
    name={modello incrementale},
    text=modello incrementale,
    sort=modello incrementale,
    description={è un modello di sviluppo software che prevede la consegna di funzionalità in maniera incrementale, cioè il prodotto finale viene 
    sviluppato attraverso una serie di rilasci parziali}
}
\newglossaryentry{sprint}
{
    name={sprint},
    text=sprint,
    sort=sprint,
    description={è un periodo di tempo breve, della durata di una o due settimane, in cui viene sviluppata 
    una funzionalità del prodotto finale}
}
\newglossaryentry{Product Backlog}
{
    name={Product Backlog},
    text=Product Backlog,
    sort=Product Backlog,
    description={è un elenco ordinato di requisiti che rappresentano le funzionalità del prodotto finale}
}
\newglossaryentry{agile}
{
    name={agile},
    text=agile,
    sort=agile,
    description={è un insieme di metodi di sviluppo software che si basano su un approccio iterativo e incrementale. 
    L'obiettivo principale dei metodi agili è quello di fornire risultati di alta qualità in modo rapido ed efficiente, consentendo ai team di adattarsi ai cambiamenti delle specifiche o dei requisiti durante il processo di sviluppo.}
}
\newglossaryentry{board}
{
    name={board},
    text=board,
    sort=board,
    description={è una bacheca virtuale che permette di visualizzare le attività da svolgere, quelle in corso e quelle completate}
}
\newglossaryentry{task}
{
    name={task},
    text=task,
    sort=task,
    description={è un compito che deve essere svolto per portare a termine un'attività più grande}
}
\newglossaryentry{container}
{
    name={container},
    text=container,
    sort=container,
    description={è un'unità software standard che raggruppa il codice e tutte le sue dipendenze in modo da poter essere eseguito in modo affidabile e veloce in qualsiasi ambiente}
}
\newglossaryentry{virtual machine}
{
    name={virtual machine},
    text=virtual machine,
    sort=virtual machine,
    description={è un ambiente computazionale autonomo e isolato che opera come una macchina fisica separata, ma è ospitato all'interno di un sistema operativo o di un altro ambiente hardware}
}
\newglossaryentry{on-promise}
{
    name={on-promise},
    text=on-promise,
    sort=on-promise,
    description={è un modello di distribuzione software in cui l'applicazione viene ospitata sul server del cliente}
}
\newglossaryentry{Docker}
{
    name={Docker},
    text=Docker,
    sort=Docker,
    description={è un progetto open-source che automatizza il deployment di applicazioni all'interno di container software}
}
\newglossaryentry{message broker}
{
    name={message broker},
    text=message broker,
    sort=message broker,
    description={è il componente intermediario che permette d' inviare e ricevere messaggi da più sorgenti verso più destinazioni. Un message broker facilita lo scambio dei messaggi 
    tra le componenti di un sistema distribuito, consentendo di comunicare in modo asincrono e disaccoppiato}
}
\newglossaryentry{streaming di eventi}
{
    name={streaming di eventi},
    text=streaming di eventi,
    sort=streaming di eventi,
    description={è una pratica di acquisizione dei dati in tempo reale da fonti di eventi come database, flussi
    di eventi; memorizzando tutto ciò per un recupero futuro di tali informazioni, reagendo a
    flussi di eventi in tempo reale}
}
\newglossaryentry{fault tolerance}
{
    name={fault tolerance},
    text=fault tolerance,
    sort=fault tolerance,
    description={è la capacità di un sistema di continuare a operare anche in caso di malfunzionamenti}
}
\newglossaryentry{open source}
{
    name={open source},
    text=open source,
    sort=open source,
    description={è un modello di sviluppo del software basato sulla condivisione del codice sorgente, che permette a chiunque di leggere, studiare, modificare e distribuire il software stesso o parte di esso}
}
\newglossaryentry{Apache Software Foundation}
{
    name={Apache Software Foundation},
    text=Apache Software Foundation,
    sort=Apache Software Foundation,
    description={è un'organizzazione non profit che supporta lo sviluppo di progetti open source}
}
\newglossaryentry{topic}
{
    name={topic},
    text=topic,
    sort=topic,
    description={è un canale o area di comunicazione a cui vengono inviati e ricevuti messaggi categorizzati}
}
\newglossaryentry{log}
{
    name={log},
    text=log,
    sort=log,
    description={è un file che registra gli eventi che si verificano durante l'esecuzione di un sistema o di un'applicazione}
}
\newglossaryentry{compattazione}
{
    name={compattazione},
    text=compattazione,
    sort=compattazione,
    description={è una strategia che si concentra sulla rimozione efficiente dei duplicati nei log dei topic, mantenendo solo la versione più recente di ciascun messaggio con una determinata chiave}
}
\newglossaryentry{cluster}
{
    name={cluster},
    text=cluster,
    sort=cluster,
    description={è un insieme di elaboratori connessi tra loro che lavorano in parallelo per eseguire un compito comune}
}
\newglossaryentry{injection}
{
    name={injection},
    text=injection,
    sort=injection,
    description={è la tecnica attraverso cui si inseriscono dati da un sistema esterno ad un altro}
}
\newglossaryentry{hands-on}
{
    name={hands-on},
    text=hands-on,
    sort=hands-on,
    description={è un'attività pratica che permette di sperimentare quanto appena appreso}
}
\newglossaryentry{volumi}
{
    name={volumi},
    text=volumi,
    sort=volumi,
    description={sono un'unità di archiviazione che possono essere montati e utilizzati da un container}
}
\newglossaryentry{immagini Docker}
{
    name={immagini Docker},
    text=immagini Docker,
    sort=immagini Docker,
    description={è un pacchetto software leggero, autonomo ed eseguibile che include tutto il necessario per eseguire un'applicazione}
}
\newglossaryentry{enviroment}
{
    name={enviroment},
    text=enviroment,
    sort=enviroment,
    description={è un insieme di variabili d'ambiente che definiscono il comportamento di un processo, le risorse utilizzabili e le informazioni di configurazione}
}
\newglossaryentry{Docker network}
{
    name={Docker network},
    text=Docker network,
    sort=Docker network,
    description={rappresenta un'interfaccia di rete virtuale che permette ai container di comunicare tra loro e con il mondo esterno}
}
\newglossaryentry{Kafka-Python}
{
    name={Kafka-Python},
    text=Kafka-Python,
    sort=Kafka-Python,
    description={è una libreria Python che consente agli sviluppatori di interagire con il sistema di messaggistica distribuita Apache Kafka utilizzando il linguaggio di programmazione Python; viene ampiamente utilizzato per la costruzione di sistemi di streaming o dove è necessario gestire grandi quantità di dati in modo scalabile e affidabile}
}
\newglossaryentry{datasource}
{
    name={datasource},
    text=datasource,
    sort=datasource,
    description={in Apache Druid è un'astrazione rappresenta un insieme di dati, organizzati in tabelle, che possono essere analizzati e interrogati}
}

\newglossaryentry{project management}
{
    name={project management},
    text=project management,
    sort=project management,
    description={è l'insieme di pratiche, metodi, processi e strumenti utilizzati per pianificare, organizzare, eseguire, monitorare e controllare le attività necessarie per raggiungere gli obiettivi di un progetto in modo efficace ed efficiente}
}
\newglossaryentry{Faker}
{
    name={Faker},
    text=Faker,
    sort=Faker,
    description={è una libreria Python che permette di generare in modo casuale una varietà di dati, come nomi, indirizzi, numeri di telefono, indirizzi email e molti altro}
}