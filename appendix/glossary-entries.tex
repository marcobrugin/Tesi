% Acronyms
\newacronym[description={\glslink{ohsg}{Occupational Health and Safety}}]
    {ohs}{OH\&S}{Occupational Health and Safety}
\newacronym[description={\glslink{sgsig}{Sistema Gestione Sicurezza Informazioni}}]
    {sgsi}{SGSI}{Sistema Gestione Sicurezza Informazioni}
\newacronym[description={\glslink{ictg}{Information and Communication Technology}}]
    {ict}{ICT}{Information and Communication Technology}
\newacronym[description={\glslink{olapg}{Online Analytical Processing}}]
    {olap}{OLAP}{Online Analytical Processing}
\newglossaryentry{olapg} {
    name=\glslink{olap}{OLAP},
    text=Online Analytical Processing,
    sort=olap,
    description={L'Online Analytical Processing è un insieme di metodi finalizzato a effettuare analisi rapide e approfondite su grandi volumi di dati, provenienti da uno o piu sorgenti, per prendere decisioni a riguardo}
}
\newglossaryentry{ictg} {
    name=\glslink{ict}{ICT},
    text=Information and Communication Technology,
    sort=ict,
    description={ICT è un termine generico che indica tutte le tecnologie che riguardano la trasmissione, la ricezione e l'elaborazione di informazioni sotto forma di segnali elettronici o elettromagnetici}
}
\newglossaryentry{sgsig} {
    name=\glslink{sgsi}{SGSI},
    text=Sistema Gestione Sicurezza Informazioni,
    sort=sgsi,
    description={Lo SGSI è un sistema di gestione che permette di gestire in modo strutturato la sicurezza delle informazioni aziendali}
}
\newglossaryentry{ohsg} {
    name=\glslink{ohs}{OH\&S},
    text=Occupational Health and Safety,
    sort=OH\&S,
    description={L'Occupational Health and Safety è un sistema di gestione che permette di gestire in modo strutturato la salute e sicurezza dei lavoratori}
}
\newglossaryentry{Data Pipeline}{
    name= \glslink{Data Pipeline}{Data Pipeline},
    text=Data Pipeline,
    sort=Data Pipeline,
    description={Una Data Pipeline è un insieme di operazioni che permettono di trasformare e analizzare i dati in modo da renderli pronti per l'archiviazione}
}
\newglossaryentry{Data Processing}
{
    name=\glslink{Data Processing}{Data Processing},
    text=Data Processing,
    sort=Data Processing,
    description={Il Data Processing (elaborazione dati) si riferisce alla manipolazione, trasformazione e analisi di dati grezzi al fine di ottenere informazioni significative e approfondite. Comprende una serie di passaggi che permettono di convertire dati non strutturati in forme più utili e che facilitano la loro elaborazione}
}
\newglossaryentry{Data Analytics}
{
    name=\glslink{Data Analytics}{Data Analytics},
    text=Data Analytics,
    sort=Data Analytics,
    description={Il Data Analytics (analisi dati) è il processo di esaminare i dati per trarne conclusioni sull'informazione che contengono}
}
\newglossaryentry{Business Innovation}
{
    name=\glslink{Business Innovation}{Business Innovation},
    text=Business Innovation,
    sort=Business Innovation,
    description={La Business Innovation è un processo che permette d' introdurre nuovi metodi, idee, prodotti e servizi per migliorare l'efficienza, la produttività e la competitività di un'organizzazione}
}
\newglossaryentry{Middleware}
{
    name=\glslink{Middleware}{Middleware},
    text=Middleware,
    sort=Middleware,
    description={Il Middleware è un software che si interpone tra un sistema operativo e le applicazioni che vengono eseguite al di sopra. Il suo scopo è quello di facilitare lo sviluppo di applicazioni e di nascondere la complessità del sistema operativo sottostante}
}
\newglossaryentry{repository}
{
    name=\glslink{repository}{repository},
    text=repository,
    sort=repository,
    description={Un repository è un ambiente di archiviazione centralizzato in cui vengono conservati e gestiti i dati}
}
\newglossaryentry{metadati}
{
    name=\glslink{metadati}{metadati},
    text=metadati,
    sort=metadati,
    description={I metadati sono dati che descrivono altri dati. Sono utilizzati per descrivere le caratteristiche dei dati e per facilitarne la ricerca e l'organizzazione}
}
\newglossaryentry{working directory}
{
    name=\glslink{working directory}{working directory},
    text=working directory,
    sort=working directory,
    description={La working directory è la directory di lavoro corrente}
}
\newglossaryentry{Scrum}
{
    name=\glslink{Scrum}{Scrum},
    text=Scrum,
    sort=Scrum,
    description={Scrum è un framework agile per la gestione del ciclo di sviluppo del software}
}
\newglossaryentry{modello incrementale}
{
    name=\glslink{modello incrementale}{modello incrementale},
    text=modello incrementale,
    sort=modello incrementale,
    description={Il modello incrementale è un modello di sviluppo software che prevede la consegna di funzionalità in maniera incrementale, cioè il prodotto finale viene 
    sviluppato attraverso una serie di rilasci parziali}
}
\newglossaryentry{sprint}
{
    name=\glslink{sprint}{sprint},
    text=sprint,
    sort=sprint,
    description={Lo sprint è un periodo di tempo breve, della durata di una o due settimane, in cui viene sviluppata 
    una funzionalità del prodotto finale}
}
\newglossaryentry{Product Backlog}
{
    name=\glslink{Product Backlog}{Product Backlog},
    text=Product Backlog,
    sort=Product Backlog,
    description={Il Product Backlog è un elenco ordinato di requisiti che rappresentano le funzionalità del prodotto finale}
}
\newglossaryentry{agile}
{
    name=\glslink{agile}{agile},
    text=agile,
    sort=agile,
    description={Il termine agile si riferisce a un insieme di metodi di sviluppo software che si basano su un approccio iterativo e incrementale. 
    L'obiettivo principale dei metodi agili è quello di fornire risultati di alta qualità in modo rapido ed efficiente, consentendo ai team di adattarsi ai cambiamenti delle specifiche o dei requisiti durante il processo di sviluppo.}
}
\newglossaryentry{board}
{
    name=\glslink{board}{board},
    text=board,
    sort=board,
    description={La board è una bacheca virtuale che permette di visualizzare le attività da svolgere, quelle in corso e quelle completate}
}
\newglossaryentry{task}
{
    name=\glslink{task}{task},
    text=task,
    sort=task,
    description={Il task è un'attività che deve essere svolta}
}


