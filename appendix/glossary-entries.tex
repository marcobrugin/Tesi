% Acronyms
\newacronym[description={\glslink{ohsg}{Occupational Health and Safety}}]
    {ohs}{OH\&S}{Occupational Health and Safety}
\newacronym[description={\glslink{sgsig}{Sistema Gestione Sicurezza Informazioni}}]
    {sgsi}{SGSI}{Sistema di Gestione della Sicurezza delle Informazioni}
\newacronym[description={\glslink{ictg}{Information and Communication Technology}}]
    {ict}{ICT}{Information and Communication Technology}
\newacronym[description={\glslink{olapg}{Online Analytical Processing}}]
    {olap}{OLAP}{Online Analytical Processing}
\newacronym[description={\glslink{tcpg}{Transmission Control Protocol}}]
    {tcp}{TCP}{Transmission Control Protocol}
\newacronym[description={\glslink{apig}{Application Programming Interface}}]
    {api}{API}{Application Programming Interface}
\newacronym[description={{Event Driven Architecture}}]
    {eda}{EDA}{Event Driven Architecture}
\newacronym[description={\glslink{jsong}{JavaScript Object Notation}}]
    {json}{JSON}{JavaScript Object Notation}
\newacronym[description={\glslink{gplg}{General Public License}}]
    {gpl}{GPL}{General Public License}
    \newglossaryentry{gplg} {
        name=\glslink{gpl}{GPL},
        text=General Public License,
        sort=GPL,
        description={in informatica con il termine di licenza \textit{Gnu Public License} si intende  
        una licenza di software libera e \textit{open source}, sviluppata dalla \textit{Free Software Foundation}, finalizzata a stabilire i termini e le condizioni 
        di utilizzo, copia, modifica e ridistribuzione del software o di un suo derivato}
    }
    \newglossaryentry{jsong}
    {
        name={JSON},
        text=JavaScript Object Notation,
        sort=JSON,
        description={è l'acronimo di \textit{JavaScript Object Notation}, un formato di scrittura leggero basato su testo per rappresentare dati strutturati. 
        È comunemente utilizzato per trasmettere dati indipendentemente dal linguaggio di programmazione utilizzato. \textit{JSON} presenta una sintassi semplice e comprensibile anche agli 
        esseri umani, attraverso la quale i dati vengono rappresentati in coppie \textbf{chiave-valore}, con la possibilità di poter includere 
        oggetti, array, numeri, stringhe, booleani e valori null}
    }
    \newglossaryentry{apig}
{
    name={API},
    text=Application Programming Interface,
    sort=API,
    description={in informatica con il termine di \textit{Application Programming Interface} si intende un insieme di interfacce di 
    programmazione, regole, protocolli e strumenti che consentono a diverse applicazioni di comunicare tra loro. La finalità delle 
    \textbf{API} è quella di definire un'astrazione tra hardware e programmatore o software a basso e alto livello, in modo da facilitare
    la programmazione e l'interazione tra i diversi componenti di un sistema}
}
\newglossaryentry{tcpg} {
    name=\glslink{tcp}{TCP},
    text=Transmission Control Protocol,
    sort=TCP,
    description={è l'acronimo di \textit{Transmission Control Protocol}, uno dei principali protocolli di rete del \textit{TCP/IP}. Si tratta di un 
    protocollo al livello di trasporto, affidabile e orientato alla connessione. Viene ampiamente utilizzato per applicazioni 
    che richiedono comunicazioni affidabili e coerenti}
}
\newglossaryentry{olapg} {
    name=\glslink{olap}{OLAP},
    text=Online Analytical Processing,
    sort=olap,
    description={è l'acronimo di \textit{Online Analytical Processing}, un insieme di tecniche e strumenti che permettono di analizzare i dati in modo 
    interattivo da più prospettive. Lo scopo principale di tali tecniche è quello di fornire, attraverso aggregazioni e compattazioni dei dati, informazioni 
    significative e approfondite finalizzate a supportare i processi decisionali}
}
\newglossaryentry{ictg} {
    name=\glslink{ict}{ICT},
    text=Information and Communication Technology,
    sort=ict,
    description={in informatica, è l'acronimo di \textit{Information and Communication Technology}, un'ampia gamma di tecnologie che riguardano la raccolta, 
    elaborazione, trasmissione e condivisione di dati e informazioni}
}
\newglossaryentry{sgsig} {
    name=\glslink{sgsi}{SGSI},
    text=Sistema Gestione Sicurezza Informazioni,
    sort=sgsi,
    description={in informatica, è l'acronimo di \textit{Sistema di Gestione della Sicurezza delle Informazioni}, un insieme coordinato di politiche, procedure, strutture di un'organizzazione finalizzate a 
    gestire e proteggere informazioni sensibili e dati critici. Lo scopo principale di tale sistema è garantire la riservatezza, l'integrità 
    e la disponibilità delle informazioni aziendali}
}
\newglossaryentry{ohsg} {
    name=\glslink{ohs}{OH\&S},
    text=Occupational Health and Safety,
    sort=OH\&S,
    description={in informatica, è l'acronimo di \textit{Occupational Health and Safety}, un insieme di norme e procedure che si concentrano 
    sulla protezione della salute, della sicurezza e del benessere dei lavoratori all'interno di un ambiente di lavoro. Ha come obiettivi 
    fondamentali la prevenzione da incendi, incidenti e infortuni sul lavoro, garantendo un ambiente di lavoro sicuro e salubre}
}
\newglossaryentry{Data Pipeline}{
    name={Data Pipeline},
    text=Data Pipeline,
    sort=Data Pipeline,
    description={in ambito di analisi dati, tale termine si riferisce a un flusso strutturato e automatizzato 
    di dati da una o più sorgenti a una o più destinazioni, attraverso una serie di passaggi di elaborazione e trasformazione}
}
\newglossaryentry{Data Processing}
{
    name={Data Processing},
    text=Data Processing,
    sort=Data Processing,
    description={\textit{Data Processing} si riferisce al processo di acquisizione, trasformazione ed elaborazione dati al fine di 
    ottenere informazioni significative. È una tecnica utilizzata in vari settori dall'analisi aziendale alla ricerca scientifica}
}
\newglossaryentry{Data Analytics}
{
    name={Data Analytics},
    text=Data Analytics,
    sort=Data Analytics,
    description={con \textit{Data Analytics} si intende il processo di pulizia, trasformazione e analisi di grandi moli di dati, finalizzato
    a ottenere informazioni significative e prendere decisioni aziendali più consapevoli}
}
\newglossaryentry{Business Innovation}
{
    name={Business Innovation},
    text=Business Innovation,
    sort=Business Innovation,
    description={in ambito aziendale, con \textit{Business Innovation} ci si riferisce alla creazione e implementazione di nuovi processi, prodotti e servizi
    che portano miglioramenti significativi all'organizzazione aziendale. La \textit{Business Innovation} ha lo scopo di portare 
    maggiore valore per il cliente, migliore adattabilità ai cambiamenti del mercato e migliore efficienza operativa}
}
\newglossaryentry{Middleware}
{
    name={Middleware},
    text=Middleware,
    sort=Middleware,
    description={in informatica, con \textit{Middleware} ci si riferisce a un insieme di software che fungono da intermediari tra diverse applicazioni 
    e sistemi, consentendo la comunicazione e lo scambio dati in modo efficiente. Il \textit{Middleware} fornisce servizi e funzionalità comuni che 
    semplificano lo sviluppo e l'integrazione di applicazioni eterogenee, facilitando flussi di informazioni tra sistemi diversi}
}
\newglossaryentry{repository}
{
    name={repository},
    text=repository,
    sort=repository,
    description={in informatica, un \textit{repository} è uno spazio o a una directory in cui archiviare e organizzare
    file, documenti e altre risorse legate allo sviluppo, gestione e monitoraggio di un progetto software}
}
\newglossaryentry{metadati}
{
    name={metadati},
    text=metadati,
    sort=metadati,
    description={i \textit{metadati} sono informazioni descrittive che forniscono l'origine, dettagli e contesto su altre risorse. Vengono ampiamente utilizzati 
    nella organizzazione, ricerca e gestione delle informazioni} 
}
\newglossaryentry{working directory}
{
    name={working directory},
    text=working directory,
    sort=working directory,
    description={in informatica, la \textit{working directopry} è la directory corrente in cui un'applicazione o un utente sta lavorando}
}
\newglossaryentry{Scrum}
{
    name={Scrum},
    text=Scrum,
    sort=Scrum,
    description={nel contesto dello sviluppo software, \textit{Scrum} è un framework di gestione e dello sviluppo agile. Si basa su principi quali trasparenza, ispezione 
    e adattamento, promuovendo il lavoro collaborativo e iterativo per raggiungere obiettivi specifici. Ha lo scopo di fornire un approccio 
    flessibile e adattabile allo sviluppo di prodotti complessi}
}
\newglossaryentry{modello incrementale}
{
    name={modello incrementale},
    text=modello incrementale,
    sort=modello incrementale,
    description={un \textit{modello incrementale} è un modello di sviluppo che prevede la consegna di funzionalità in modo incrementale, creando 
    il prodotto finale attraverso rilasci successivi e parziali, ottenendo così feedback tempestivi e frequenti}
}
\newglossaryentry{sprint}
{
    name={sprint},
    text=sprint,
    sort=sprint,
    description={nel contesto del framework \textit{Scrum}, con \textit{sprint} si intende un periodo di tempo predeterminato durante il quale il team di sviluppo lavora su una serie di attività e obiettivi specifici. In genere 
    gli \textit{sprint} hanno una durata media di due o quattro settimane e sono progettati per promuovere un consegna iterattiva 
    del prodotto finale}
}
\newglossaryentry{Product Backlog}
{
    name={Product Backlog},
    text=Product Backlog,
    sort=Product Backlog,
    description={nel contesto del framework \textit{Scrum}, il \textit{Product Backlog} è una lista dinamica e prioritizzata di tutte le funzionalità, le caratteristiche, le attività e gli elementi che devono essere sviluppati e implementati in un prodotto software. È
    il punto di partenza per la pianificazione degli \textit{sprint}}
}
\newglossaryentry{agile}
{
    name={agile},
    text=agile,
    sort=agile,
    description={nel contesto dello sviluppo software, con agile si intende un insieme di metodi di sviluppo che si basano su un approccio iterativo e incrementale. 
    I metodi agili pongono l'accento sulla risposta rapida ai cambiamenti, sulla consegna frequente di valore e sulla creazione di un ambiente di lavoro collaborativo e adattabile.
    \\I più noti sono \textit{Scrum}, Kanban, Extreme Programming (XP) e Feature-Driven Development (FDD)}
}
\newglossaryentry{board}
{
    name={board},
    text=board,
    sort=board,
    description={nel contesto dello sviluppo software, una \textit{board} è una bacheca virtuale riportante 
    una istantanea del progetto, con attività svolte, da svolgere e in corso di svolgimento}
}
\newglossaryentry{task}
{
    name={task},
    text=task,
    sort=task,
    description={all'interno della gestione di un progetto software, un \textit{task} è un'attività che deve essere svolta per raggiungere un obiettivo specifico}
}
\newglossaryentry{container}
{
    name={container},
    text=container,
    sort=container,
    description={nel contesto della virtualizzazione, un \textit{container} è un'unità software standard che raggruppa il codice e tutte 
    dipendenze con modalità comuni per essere eseguite ovunque. Per raggiungere tale scopo i \textit{container} usufruiscono di una forma 
    di virtualizzazione (nel caso del \textit{Kernel Linux} i \textit{namespace} e i \textit{cgroup}), per isolare i processi in esecuzione 
    e controllare le risorse che questi ultimi usufruiscono.
    I \textit{container} nascono per essere piccoli, veloci e portabili, non hanno bisogno d'includere in ogni istanza un sistema operativo,
    sfruttando invece le funzioni e le risorse del sistema operativo host}
}
\newglossaryentry{virtual machine}
{
    name={virtual machine},
    text=virtual machine,
    sort=virtual machine,
    description={nel contesto della virtualizzazione, una \textit{virtual machine} è un ambiente software isolato e autonomo che emula un elaboratore fisico completo.
    In ogni istanza di una \textit{virtual machine} è presente un sistema operativo completo, utilizzando le risorse 
    fisiche del sistema host. L'isolamento tra le varie \textit{virtual machine} è garantito da un 
    ulteriore componente software, chiamato \textit{hypervisor}, che si occupa di assegnare le risorse fisiche del sistema host tra le varie \textit{virtual machine} e di gestire le comunicazioni tra esse, se permessa.\\
    A differenza dei \textit{container}, le \textit{virtual machine} sono più grandi, più lente e meno portabili, in quanto dipendenti dall'\textit{hipervisor} utilizzato}
}
\newglossaryentry{on-promise}
{
    name={on-promise},
    text=on-promise,
    sort=on-promise,
    description={nella distribuzione del software, una tecnologia \textit{on-promise} rappresenta
    un modello in cui i servizi sono ospitati e gestiti all'interno di una 
    organizzazione, che detiene e controlla le risorse hardware e software}
}
\newglossaryentry{Docker}
{
    name={Docker},
    text=Docker,
    sort=Docker,
    description={è una piattaforma \textit{open-source} che consente di creare, gestire applicazioni in \textit{container} e distribuirle in modo semplice e veloce.
    \textit{Docker} nasce nel 2010, quando Solomon Hykes sviluppò un progetto interno chiamato \textbf{dotCloud} presso la sua azienda di produzione software. In seguito 
    nel 2013 \textit{Docker} è stato rilasciato come progetto \textit{open-source}, rivoluzionando la modalità di sviluppo delle applicazioni}
}
\newglossaryentry{message broker}
{
    name={message broker},
    text=message broker,
    sort=message broker,
    description={in informatica, un \textit{message broker} è  il componente intermediario che permette d' inviare e ricevere messaggi da più sorgenti verso più destinazioni,
    Un \textit{message broker} facilita lo scambio dei messaggi 
    tra le componenti di un sistema distribuito, consentendo di comunicare in modo asincrono e disaccoppiato.
    Viene particolarmente utilizzato in sistemi di monitoraggio e applicazioni \textbf{IoT}}
}
\newglossaryentry{streaming di eventi}
{
    name={streaming di eventi},
    text=streaming di eventi,
    sort=streaming di eventi,
    description={in informatica, lo \textit{streaming di eventi} rappresenta una pratica di acquisizione dati in tempo reale da fonti diverse, 
    tramite l'invio di eventi. Tale approccio consente l'elaborazione e l'analisi di tali dati, nel momento in cui si verificano,
    consentendo di ottenere informazioni significative e tempestive}
}
\newglossaryentry{fault tolerance}
{
    name={fault tolerance},
    text=fault tolerance,
    sort=fault tolerance,
    description={con \textit{fault tolerance} si intende la capacità di un sistema di operare in modo corretto e fornire 
    i servizi anche in presenza di guasti o malfunzionamenti}
}
\newglossaryentry{open source}
{
    name={open source},
    text=open source,
    sort=open source,
    description={con il termine \textit{open source}, in informatica, ci si riferisce a un tipo di sviluppo software in cui il codice sorgente è reso pubblico e accessibile a tutti,
    consentendo a chiunque di leggerlo, studiarlo, modificarlo e ridistribuirlo.
    Un software \textit{open source} è caratterizzato da uno sviluppo collaborativo e trasparente.\\
    Un esempio molto noto di software \textit{open source} è il \textit{kernel Linux}}
}
\newglossaryentry{Apache Software Foundation}
{
    name={Apache Software Foundation},
    text=Apache Software Foundation,
    sort=Apache Software Foundation,
    description={è un'organizzazione senza scopo di lucro dedicata allo sviluppo e alla promozione di software \textit{open source}.\\
    Fondata nel 1999, oggi è diventata un punto di riferimento per la comunità \textit{open source}, con oltre 350 progetti supportati}
}
\newglossaryentry{topic}
{
    name={topic},
    text=topic,
    sort=topic,
    description={con \textit{topic}, in un sistema di messaggistica, si intende una categoria a cui i componenti del sistema possono inviare e 
    ricevere messaggi in modo asincrono. I \textit{topic} sono utilizzati per organizzare, classificare e filtrare i messaggi}
}
\newglossaryentry{log}
{
    name={log},
    text=log,
    sort=log,
    description={un \textit{log}, in informatica, rappresenta un file o un insieme di dati, che registra eventi, azioni e informazioni rilevanti, 
    che si verificano all'interno di un sistema}
}
\newglossaryentry{compattazione}
{
    name={compattazione},
    text=compattazione,
    sort=compattazione,
    description={nel contesto dell'analisi dati, la compattazione è una strategia che si concentra sulla rimozione efficiente dei duplicati nei \textit{log} dei \textit{topic}, mantenendo solo la versione più recente di ciascun messaggio con una determinata chiave}
}
\newglossaryentry{cluster}
{
    name={cluster},
    text=cluster,
    sort=cluster,
    description={in informatica, un \textit{cluster} rappresenta un insieme di elaboratori connessi tra di loro tramite una rete, configurati 
    per lavorare in modo collaborativo e coordinato, al fine di eseguire un compito comune}
}
\newglossaryentry{injection}
{
    name={injection},
    text=injection,
    sort=injection,
    description={nel contesto dell'analisi dati, l'injection è  una pratica attraverso la quale dei dati 
    vengono inseriti da un sistema ad un altro}
}
\newglossaryentry{hands-on}
{
    name={hands-on},
    text=hands-on,
    sort=hands-on,
    description={in ambito di formazione, con \textit{hands-on} si intende un'attività pratica che permetta di sperimentare 
    concetti e consolidare conoscenze teoriche appena apprese}
}
\newglossaryentry{volumi}
{
    name={volumi},
    text=volumi,
    sort=volumi,
    description={in ambito di virtualizzazione, legata ai \textit{container}, con \textit{volume} si intende una risorsa che permette di gestire 
    e persistere dati tra il sistema host e i \textit{container}. Inoltre i \textit{volumi} permettono di condividere e separare 
    dati tra i \textit{container}}
}
\newglossaryentry{immagini Docker}
{
    name={immagini Docker},
    text=immagini Docker,
    sort=immagini Docker,
    description={in ambito di virtualizzazione, legata ai \textit{container Docker}, le immagini sono pacchetti leggeri e autosufficienti,
    contengono tutto il necessario per eseguire un' applicazione. Sono le componenti essenziali dei \textit{container Docker} e consentono di 
    creare e gestire applicazioni in modo semplice e veloce}
}
\newglossaryentry{enviroment}
{
    name={enviroment},
    text=enviroment,
    sort=enviroment,
    description={in informatica, con \textit{enviroment}, si intende un insieme di variabili d'ambiente che definiscono il comportamento di un processo, le risorse utilizzabili e le informazioni di configurazione}
}
\newglossaryentry{Docker network}
{
    name={Docker network},
    text=Docker network,
    sort=Docker network,
    description={in ambito \textit{Docker}, una \textit{Docker network} rappresenta un'interfaccia di rete virtuale, che permette ai \textit{container}, che la dichiarano nel proprio file di configurazione, di comunicare tra loro e, se permesso, con il sistema host}
}
\newglossaryentry{Kafka-Python}
{
    name={Kafka-Python},
    text=Kafka-Python,
    sort=Kafka-Python,
    description={\textit{Kafka-Python} è una libreria \textbf{Python} che consente agli sviluppatori d'interagire con il sistema di messaggistica \textbf{Apache Kafka}, utilizzando il linguaggio di programmazione \textbf{Python}. Tale libreria viene ampiamente utilizzato per la costruzione di sistemi di streaming o dove è necessario gestire grandi quantità di dati in modo scalabile e affidabile}
}
\newglossaryentry{datasource}
{
    name={datasource},
    text=datasource,
    sort=datasource,
    description={in Apache Druid, una \textit{datasource} è un'astrazione che rappresenta un insieme di dati, organizzati in tabelle, che possono essere analizzati e interrogati}
}

\newglossaryentry{project management}
{
    name={project management},
    text=project management,
    sort=project management,
    description={nel contesto di gestione di progetto, con \textit{project management} si indica l'insieme di pratiche, metodi, processi e strumenti utilizzati per pianificare, organizzare, eseguire, monitorare e controllare le attività necessarie per raggiungere gli obiettivi di un progetto in modo efficace ed efficiente}
}
\newglossaryentry{Faker}
{
    name={Faker},
    text=Faker,
    sort=Faker,
    description={è una libreria Python che permette di generare in modo casuale una varietà di dati, come nomi, indirizzi, numeri di telefono, indirizzi email e molto altro}
}